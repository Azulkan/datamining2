\documentclass[a4paper,12pt]{article}

\usepackage{ucs}
\usepackage[utf8]{inputenc}
\usepackage[english]{babel}
\usepackage{textcomp}
\usepackage{bm}
\usepackage{xspace}
\usepackage{algpseudocode}
\usepackage{algorithm}

\usepackage{graphicx}
\usepackage{authblk}
\usepackage{textcomp}
\usepackage[symbol*]{footmisc}
\usepackage{amsmath}
\usepackage{amssymb}
\usepackage{amsfonts}
\usepackage{mathtools}
\usepackage{mathrsfs}
\usepackage{bm}
\usepackage{hyperref}
\usepackage{paralist}
\usepackage{upgreek}
\usepackage[tight]{subfigure}
\hypersetup{
  colorlinks,
  citecolor=red,
  filecolor=black,
  linkcolor=blue,
  urlcolor=black
}

\usepackage[hmargin=20mm,vmargin={20mm,30mm}]{geometry}

\usepackage[section]{placeins}
\bibliographystyle{plain}

\usepackage{tikz}
\usetikzlibrary{shapes,arrows,patterns,calc}
\pgfdeclarelayer{background}
\pgfdeclarelayer{foreground}
\pgfsetlayers{background,main,foreground}
\newcommand{\HRule}[1]{\rule{\linewidth}{#1}}
\newcommand\Mycomb[2][n]{\prescript{#1\mkern-0.5mu}{}C_{#2}}

\title{\HRule{1.5mm} \\[0.4cm] \textbf{Hands on Lab 1\\
{\small A Gentle and Practical Introduction to Neural Networks}}\\ \HRule{0.5mm} \\[0.4cm]}

\author[1]{First name Family name \footnote{\texttt{fn.fn@insa-rouen.fr}}}

\affil[1]{Normandie Univ, UNIROUEN, INSA Rouen, LITIS, 76000 Rouen, France.}


\date{March, XX$^{th}$.2017}

\input{./notations}


\begin{document}
\maketitle 

\begin{abstract}
  You will perform in this lab the classification task using neural network models. First, you will use existing libraries where everything is ready to use (Keras\footnote{\url{https://keras.io/}}). Next, you will build your own neural networks and train them using Theano\footnote{\url{http://deeplearning.net/software/theano/}}. Provide the result that you find.
  
  \textcolor{red}{Keep your code clean and modular as possible. You will re-use it later.}
  
  \textcolor{red}{At each Exercise/question, describe how did you proceed to solve the problem.}
\end{abstract}
\section{Exercise 1}
Neural networks for classification over Mnist dataset.
\begin{enumerate}
\item Install Keras. How?
\item Download Mnist dataset from here: \url{https://sbelharbi.github.io/otherdocs/teaching2016-2017/ASI4/datasets/mnist.pkl.gz}
\item How did you download it?
\item Give a description of the dataset Mnist. (what are the pertinent information about this set?).
\item Build different neural network models (using different number of hidden layers, different number of neurons). Start from smaller networks to bigger ones.
\item What it the size of each model? (i.e. the total number of parameters)
\item Train them using few/a lot of training examples. Do you see any over-fitting?. When you select only few samples samples, how did you do it? Save the model. Compare.
\item Plot the training loss and validation loss. Describe your strategy for the learning rate. Plot the evolution of the learning rate. Use small/large learning rate. Comment. 
\item When did you stop training? why?
\item Find the best model for your problem. How did you do it?
\item Load the saved models. Evaluate them.
\item Use regularization. Is there a difference? how did you fix the regularization coefficient?
\end{enumerate}

\section{Exercise 2}
Same questions as in Exercise 1, but this time you will build your own neural networks and you will train them by yourself using Theano.

\end{document}
